% (c) 2002 Matthew Boedicker <mboedick@mboedick.org> (original author) http://mboedick.org
% (c) 2003-2007 David J. Grant <davidgrant-at-gmail.com> http://www.davidgrant.ca
% (c) 2008 Nathaniel Johnston <nathaniel@nathanieljohnston.com> http://www.nathanieljohnston.com
% (l) 2012 Arun I B <arunib@smail.iitm.ac.in> http://www.ee.iitm.ac.in/~ee10s026/
%This work is licensed under the Creative Commons Attribution-Noncommercial-Share Alike 2.5 License. To view a copy of this license, visit http://creativecommons.org/licenses/by-nc-sa/2.5/ or send a letter to Creative Commons, 543 Howard Street, 5th Floor, San Francisco, California, 94105, USA.

\documentclass[letterpaper,11pt]{article}
\newlength{\outerbordwidth}
\pagestyle{empty}
\raggedbottom
\raggedright
\usepackage[svgnames]{xcolor}
\usepackage{framed}
\usepackage{times}
\usepackage{tocloft}
\usepackage{graphicx}
\usepackage{multirow}
\usepackage[utf8]{inputenc}
\usepackage{tabularx}
\usepackage{verbatim}	%块注释
\usepackage{hyperref}%使用url

\title{Aparna-CV}
%-----------------------------------------------------------
%Edit these values as you see fit

\setlength{\outerbordwidth}{3pt}  % Width of border outside of title bars
\definecolor{shadecolor}{gray}{0.75}  % Outer background color of title bars (0 = black, 1 = white)
\definecolor{shadecolorB}{gray}{0.93}  % Inner background color of title bars


%-----------------------------------------------------------
%Margin setup

\setlength{\evensidemargin}{-0.25in}
\setlength{\headheight}{0in}
\setlength{\headsep}{0in}
\setlength{\oddsidemargin}{-0.25in}
\setlength{\paperheight}{11in}
\setlength{\paperwidth}{8.5in}
\setlength{\tabcolsep}{0in}
\setlength{\textheight}{9.5in}
\setlength{\textwidth}{7in}
\setlength{\topmargin}{-0.3in}
\setlength{\topskip}{0in}
\setlength{\voffset}{0.1in}


%-----------------------------------------------------------
%Custom commands
\newcommand{\resitem}[1]{\item #1 \vspace{-2pt}}
\newcommand{\resheading}[1]{\vspace{8pt}%{\vspace{8pt}
  \parbox{\textwidth}{\setlength{\FrameSep}{\outerbordwidth}
    \begin{shaded}
\setlength{\fboxsep}{0pt}\framebox[\textwidth][l]{\setlength{\fboxsep}{4pt}\fcolorbox{shadecolorB}{shadecolorB}{\textbf{\sffamily{\mbox{~}\makebox[6.762in][l]{\large #1} \vphantom{p\^{E}}}}}}
    \end{shaded}
  }\vspace{-5pt}%\vspace{-5pt}
}
\newcommand{\ressubheading}[4]{
\begin{tabular*}{6.5in}{l@{\cftdotfill{\cftsecdotsep}\extracolsep{\fill}}r}
		\textbf{#1} & #2 \\
		\textit{#3} & \textit{#4} \\
\end{tabular*}\vspace{-6pt}}
%-----------------------------------------------------------

\begin{document}

%-----------------------------------------------------------
%Insert IIT Madras Logo 
\begin{tabular*}{7in}{l@{\extracolsep{\fill}}r}
  & \multirow{4}{*}{\includegraphics[scale=0.95]{hjh.jpg}}\\%  & \multirow{4}{*}{\includegraphics[scale=0.19]{iitmlogo}}
  & \\
%-----------------------------------------------------------  
  \textbf{\Large Junhao Huang} & \\% $|$ EE10S0600 
  PhD Student& \\%Research Scholar
  BNU-HKBU United International College, Zhuhai, China  \\
  jhhuang\_nuaa@126.com \\
  \href{https://junhaohuang.github.io/}{Home Page}
\end{tabular*}
\\


%%%%%%%%%%%%%%%%%%%%%%%%%%%%%%
\resheading{Education}
%%%%%%%%%%%%%%%%%%%%%%%%%%%%%%

\begin{itemize}
\item
	\ressubheading{BNU-HKBU United International College}{Supervisor: Prof. Donglong Chen}{PhD student at Data Science and Technology}{Sep. 2021-now}
\begin{comment}%开始注释
	\begin{itemize}
		\resitem{Major Courses: Mathematical foundation of information security, Modern Cryptography and Application, Blockchain Application, Digital Design and Computer architecture.}
	\end{itemize}
\end{comment}%结束注释
\item
	\ressubheading{Nanjing University of Aeronautics and Astronautics}{Supervisor: Prof. Zhe Liu}{Master Degree of Cyberspace Security}{Sep. 2018-Jun. 2021}
\begin{comment}%开始注释
	\begin{itemize}
		\resitem{Major Courses: Mathematical foundation of information security, Modern Cryptography and Application, Blockchain Application, Digital Design and Computer architecture.}
	\end{itemize}
\end{comment}%结束注释

\item
	\ressubheading{Nanjing University of Aeronautics and Astronautics}{GPA: 3.7}{Bachelor Degree of Computer Science and Technology}{Sep. 2014-Jun. 2018}
\begin{comment}%开始注释
	\begin{itemize}
		\resitem{Major Courses: C/C++, Data Structure, Algorithm Design, Software Engineering, Computer Organization and Assembly Language, Operating System.}
	\end{itemize}
\end{comment}%结束注释

\end{itemize}

\begin{comment}%开始注释
\begin{itemize}
\item
	\ressubheading{My University}{My Town,}{B.Sc. Physics}{2004 - 2008}
	\begin{itemize}
		\resitem{Undergraduate Thesis: Why Electron Spins Rule}
		\resitem{Graduated with Honours and a XX.X\% average}
	\end{itemize}

\item
	\ressubheading{My High School}{Hick Town, ON}{High School Diploma}{2000 - 2004}
	\begin{itemize}
		\resitem{President of Students' Council and captain of the rugby team in senior year}
		\resitem{Graduated with a XX.X\% average}
	\end{itemize}
\end{itemize}
\end{comment}%结束注释
%%%%%%%%%%%%%%%%%%%%%%%%%%%%%%
\resheading{Research Interest}
%%%%%%%%%%%%%%%%%%%%%%%%%%%%%%
\begin{itemize}
\item
	Cryptographic Engineering, Public-key Cryptography, Lattice-based Cryptography.

\end{itemize}


%%%%%%%%%%%%%%%%%%%%%%%%%%%%%%
\resheading{Research Activities}
%%%%%%%%%%%%%%%%%%%%%%%%%%%%%%
\begin{itemize}
\item
	\ressubheading{Teaching Assistant}{Zhuhai, China}{BNU-HKBU United International College, Computer and Network Security}{Sep. 2021-Now}
\item
	\ressubheading{Research Assistant}{Whuhan, China}{Wuhan University, Cryptography and Blockchain Technology Lab}{Sep. 2019-Jan. 2020}

\end{itemize}

%%%%%%%%%%%%%%%%%%%%%%%%%%%%%%
\resheading{Publications}
%%%%%%%%%%%%%%%%%%%%%%%%%%%%%%
\textbf{- Journal Publications}
\begin{enumerate}\setlength{\itemsep}{0pt}
	\item {Improved Plantard Arithmetic for Lattice-based Cryptography,\\
	\textbf{Junhao Huang}, Jipeng Zhang, Haosong Zhao, Zhe Liu, Ray C.C. Cheung, Çetin Kaya Koç, Donglong Chen. \\
	In \textcolor{blue}{IACR Transactions on Cryptographic Hardware and Embedded Systems, 2022} (\textbf{CCF-B})}
	\item {Time-memory Trade-offs for Saber on Memory-constrained RISC-V,\\
	Jipeng Zhang, \textbf{Junhao Huang}, Zhe Liu, Sujoy Sinha Roy. \\
	In \textcolor{blue}{IEEE Transactions on Computers, 2022} (\textbf{CCF-A})}
	\item {High-Speed AVX2 Implementation of AKCN-MLWE,\\
	YANG Hao, LIU Zhe, \textbf{HUANG Jun-Hao}, SHEN Shi-Yu  ZHAO Yun-Lei. \\
	In \textcolor{blue}{Chinese Journal of Computers, 2021}}	
\end{enumerate}
\textbf{- Conference Publications}
\begin{enumerate}\setlength{\itemsep}{0pt}
	\item {Parallel Implementation of SM2 Elliptic Curve on Intel Processor with AVX2. \\\textbf{Junhao Huang}, Zhe Liu, Zhi Hu, and Johann Großschädl. \\ In \textcolor{blue}{Australasian Conference on Information Security and Privacy - ACISP 2020 } (\textbf{CCF-C})}
	\item {An Efficient and Scalable Sparse Polynomial Multiplication Accelerator for LAC on FPGA,\\
	Jipeng Zhang, Zhe Liu, Hao Yang, \textbf{Junhao Huang}, Weibin Wu. \\
	In \textcolor{blue}{IEEE International Conference on Parallel and Distributed Systems - ICPADS 2020} (\textbf{CCF-C})}
	\item {Efficient Implementation of Kyber on Mobile Devices,\\
	Lirui Zhao, Jipeng Zhang, \textbf{Junhao Huang}, Zhe Liu, Gerhard Hancke,\\
	In \textcolor{blue}{IEEE International Conference on Parallel and Distributed Systems - ICPADS 2021} (\textbf{CCF-C})}
\end{enumerate}

%%%%%%%%%%%%%%%%%%%%%%%%%%%%%%
\resheading{Reaserch Experiences}
%%%%%%%%%%%%%%%%%%%%%%%%%%%%%%
\begin{itemize}
	\item
	Sep. 2021-Apr. 2022\quad Improved Plantard Arithmetic for Lattice-based Cryptography
% \begin{comment}%开始注释
	\begin{itemize}
		\resitem{Present an improved Plantard arithmetic tailored for LBC.}
		\resitem{Obtained speed-ups for Kyber and NTTRU with 16-bit NTT on Cortex-M4.}
		\resitem{The source code has been merged into \href{https://github.com/mupq/pqm4/pull/244}{pqm4}, see PR\#244 (merged at 25th, Oct, 2022).}
	\end{itemize}

	\item
	Dec. 2020-now\quad Memory Efficient Implementation of Saber on RISC-V
% \begin{comment}%开始注释
	\begin{itemize}
		\resitem{Reduce the memory usage of Saber by using a \textbf{just-in-time} public matrix, secret, and noise generation technique.}
		\resitem{Represent the secret, and noise with a new \textbf{smaller data-type}, which reduces the size of the secret and noise.}
	\end{itemize}
	\item
	Apr. 2019-Nov. 2020\quad Accelerating ECC utilizing the Double Precision Floating-point Number on GPU
% \begin{comment}%开始注释
	\begin{itemize}
		\resitem{Implement the prime field arithmetic for the prime modulus $p=2^n-\delta$ by combining the computing power of \textbf{the fused multiply-add instruction of double-precision floating-point number} and the addition, subtraction, and shift instructions of integer number. }
		\resitem{Propose how to perform multi-precision multiplication over unreduced-form big number, which optimizes the point multiplication, especially Montgomery ladder algorithm for Montgomery curves, with the \textbf{lazy reduction technique}.}
	\end{itemize}
	\item
	Sep. 2019-Mar. 2020\quad Accelerating SM2 on GPU
% \begin{comment}%开始注释
	\begin{itemize}
		\resitem{Implement the prime field arithmetic for SM2 using the low-level PTX assembly language on GPU, which contributes to the performance of the high-level point arithmetic and cryptographic protocols of SM2.}
	\end{itemize}
	\item
	Apr. 2019-Oct. 2019\quad Parallel Implementation of SM2 Elliptic Curve with AVX2
% \begin{comment}%开始注释
	\begin{itemize}
		\resitem{Utilize SIMD AVX2 instruction set to implement 2-way SM2 prime field operations.}

		\resitem{Reschedule the (X,Y)-only Co-Z Jacobian arithmetic and perform the symmetric operations using the 2-way prime field operations}

		\resitem{Implement the Co-Z based Montgomery ladder algorithm based on the parallel Co-Z Jacobian arithmetic.}

		\resitem{The number of the 2-way prime field operations of the Co-Z Jacobian arithmetic is reduced to a half compared to the sequential implementation.}

		\resitem{The AVX2 version Co-Z based Montgomery ladder algorithm is \textbf{1.31} times faster than the X64 assembly implementation.}
	\end{itemize}
% \item 
% 	Nov. 2016-Mar. 2017\quad University Association Information Management System (APP)
% 	\begin{itemize}
% 		\resitem{An app that facilitates internal communication and management of associations, and simplifies members' participation in association activities.}
% 		\resitem{Achieve Association management, Association activities management, Association member management.}
% 		\resitem{Applied for a \textbf{Software Copyright}.}
% 	\end{itemize}
\end{itemize}



%%%%%%%%%%%%%%%%%%%%%%%%%%%%%%
\resheading{Honor Certificates}
%%%%%%%%%%%%%%%%%%%%%%%%%%%%%%
\begin{itemize}
\item 
	Nov.2019\quad   Patent for An efficient implementation of Co-Z based Montgomery ladder algorithm using AVX2, CN112367172A.
\item
	Oct. 2018\quad	Postgraduate \textbf{First prize} Scholarship
\item
	Oct. 2018\quad	\textbf{First Prize} of Academic Scholarship
\item
	Jun. 2018\quad	Software Copyright for the University Association Information Management System
\item
	Oct. 2017\quad	National Encouragement Scholarship, \textbf{Third Prize} of Outstanding Student Scholarship
\item
	Oct. 2016\quad	National Encouragement Scholarship, \textbf{Second Prize} of Outstanding Student Scholarship
\item
	Oct. 2015\quad	National Encouragement Scholarship, \textbf{First Prize} of Outstanding Student Scholarship

\end{itemize}


%%%%%%%%%%%%%%%%%%%%%%%%%%%%%%
\resheading{Professional Skills}
%%%%%%%%%%%%%%%%%%%%%%%%%%%%%%
\begin{enumerate}\setlength{\itemsep}{0pt}
	\item {Language Level: CET-4: 597, CET-6: 513, \textbf{IELTS: 7.0}}
	\item {Programming Language: C/C++, x86-64/Cortex-M4/Cortex-M3/RISC-V Assembly, AVX2 and CUDA programming, Python}
\end{enumerate}

% \begin{comment}%开始注释
%%%%%%%%%%%%%%%%%%%%%%%%%%%%%%
\resheading{Self Introduction}
%%%%%%%%%%%%%%%%%%%%%%%%%%%%%%
  \begin{center}
  \parbox{6.762in}{I have been implementing elliptic curve cryptography since I was a graduate student. I tried to implement SM2 and other elliptic curves using different languages on different platforms, i.e. C, x86-64 assembly language, AVX2 on CPU, and CUDA programming on GPU. During the 5-month exchange study at Wuhan University, Lattice-based Cryptography and Blockchain are two other research areas of my interest. Recently, I've been trying to implement Kyber on a RISC-V chip, which further expands my experiences on cryptographic engineering.}
  \end{center}
% \end{comment}%结束注释
\end{document}